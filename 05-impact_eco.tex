
\documentclass[11pt, a4paper]{book}
\input{package.tex}
\begin{document}

\setcounter{chapter}{4}

\chapter{Impact environnemental du numérique}

\section{Introduction}
L’impact du numérique sur l’environnement est multiple :
\begin{itemize}
  \item \textbf{Gaz à effet de serre (GES)},
  \item Consommation d’énergie électrique,
  \item Déchets toxiques,
  \item Utilisation de matériaux rares,
  \item Utilisation et pollution de l’eau,
  \item Atteinte à la biodiversité (faune et flore).
\end{itemize}

\begin{figure}[h!]
  \centering
  \includegraphics[width=0.7\textwidth]{images/impact-eco/image11.png}
  \caption{Chaîne de fabrication et d'utilisation d'un appareil électronique}
  \label{fig:intro-impact}
\end{figure}

\section{Gaz à effet de serre (GES)}

\subsection{Définition}
Les \textbf{gaz à effet de serre (GES)} sont des composants gazeux présents dans l'atmosphère terrestre qui piègent la chaleur émise par la Terre après qu'elle ait absorbé l'énergie solaire. 


Ce \textbf{phénomène naturel}, appelé effet de serre, est essentiel pour maintenir une température permettant la vie sur notre planète.


Cependant, lorsque les concentrations de ces gaz \textbf{augmentent de manière excessive}, en grande partie à cause des activités humaines, ils provoquent un \textbf{réchauffement climatique}, conduisant à des changements importants dans les systèmes climatiques mondiaux.




\subsection{Contexte, conséquences et objectifs}
Selon le dernier \textbf{rapport du GIEC }(Groupe Intergouvernemental d’experts sur l’évolution du climat)(2023) :
\begin{itemize}
  \item Température moyenne de la surface du globe : +1,1 °C par rapport à la période pré-industrielle,
  \item \textbf{+1,5 °C attendus dès le début des années 2030.}
\end{itemize}



\begin{figure}[h!]
  \centering
  \includegraphics[width=0.65\textwidth]{images/impact-eco/image10.png}
  \caption{Évolution des émissions de gaz à effet de serre.}
  \label{fig:ges-schema}
\end{figure}
\begin{figure}[h!]
  \centering
  \includegraphics[width=0.65\textwidth]{images/impact-eco/image3.png}
  \caption{Conséquences sur la faune du dérèglement climatique.}
  \label{fig:ges-faune}
\end{figure}

\begin{figure}[h!]
  \centering
  \includegraphics[width=0.6\textwidth]{images/impact-eco/image8.png}
  \caption{Exemples d’activités et de leurs émissions de GES.}
  \label{fig:activites}
\end{figure}


Ils en tirent la conséquence suivante : 
\begin{itemize}
  \item \textbf{Il est urgent de limiter ce réchauffement à 1,5°C et 2 °C  ce qui n'est possible qu’en réduisant drastiquement les émissions de GES.}
\end{itemize}

En 2019, la Suisse a donc décidé de ramener ses émissions de gaz à effet de serre au \textbf{zéro net d’ici 2050}.

\begin{figure}[h!]
  \centering
  \includegraphics[width=0.6\textwidth]{images/impact-eco/image7.png}
  \caption{Objectifs de réduction des émissions de GES en Suisse.}
  \label{fig:objectif-suisse}
\end{figure}

\subsection{Unités de mesure}
\subsubsection{Gaz à effet de serre}
L'émission en équivalent CO2 d’un GES est la quantité émise de dioxyde de carbone (CO2) qui provoquerait le même effet de réchauffement (forçage radiatif intégré) que le gaz mesuré.


Unité : \textbf{kgCO2eq}

\subsubsection{Consommation électrique}
\begin{itemize}
    \item Unité de l’énergie du système international (SI) : le \textbf{Joule [J]}.\\
    → Exemple besoin journalier : 2’000 Kilocalorie = 8’368’000 [J]
    \item Unité de puissance : le \textbf{Watt [W]} = 1 [J/s]\\
    → Exemple puissance d’un micro-ondes = 1’000 [W]
    \item Unité d’énergie consommée : le \textbf{Watt-heure [Wh]}. 1 [Wh] = 3’600 [J] \\
    → Exemple énergie d’une batterie de laptop : 50 [Wh] = 0,05 [kWh]
\end{itemize}




\section{Impact du numérique}
\begin{itemize}
  \item Le numérique représente environ 3--4 \% des émissions mondiales de GES,
  \item Soit l’équivalent du secteur de l’aviation civile.
\end{itemize}

\begin{figure}[h!]
  \centering
  \includegraphics[width=0.6\textwidth]{images/impact-eco/image6.png}
  \caption{Quelques chiffres français}
  \label{fig:unites}
\end{figure}




\section{Que faire ?}
\subsection{Allonger la durée de vie du matériel}
\begin{itemize}
  \item Conserver plus longtemps ses équipements,
  \item Choisir des marques éthiques et réparables,
  \item Réparer soi-même ou faire réparer,
  \item Donner ou revendre plutôt que jeter.
\end{itemize}

\subsection{Réduire la taille des données échangées et stockées}
\begin{itemize}
  \item Faire le tri dans ses données (cloud, mails, réseaux sociaux),
  \item Dégrader la qualité quand c’est pertinent,
  \item Privilégier le texte aux contenus lourds (images, vidéos),
  \item Préférer les liens aux pièces jointes.
\end{itemize}

\subsection{Optimiser l’usage du réseau}
\begin{itemize}
  \item Privilégier le stockage local au streaming,
  \item Préférer les connexions filaires au Wi-Fi, et le Wi-Fi à la 4G/5G.
\end{itemize}


\section*{Sources}
\begin{itemize}
    \item \url{https://enseigner.modulo-info.ch/enjx2/activ/emission_ges.html}
    \item \url{https://www.arcep.fr/la-regulation/grands-dossiers-thematiques-transverses/lempreinte-environnementale-du-numerique.html}
    \item \url{https://www.ecologie.gouv.fr/actualites/publication-du-6e-rapport-synthese-du-giec}
    \item \url{https://www.bafu.admin.ch/bafu/fr/home/themes/climat/info-specialistes/reduction-emissions/objectifs-reduction/objectif-2050/strategie-climatique-2050.html}
    \item \url{https://fr.wikipedia.org/wiki/%C3%89quivalent_CO2}
    \item \url{https://infos.ademe.fr/magazine-avril-2022/faits-et-chiffres/numerique-quel-impact-environnemental/}
    \item \url{https://www.europe1.fr/technologies/trois-chiffres-pour-comprendre-limmense-impact-ecologique-du-numerique-4036952}
\end{itemize}
\end{document}